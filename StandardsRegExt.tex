\documentclass[11pt,a4paper]{ivoa}
\input tthdefs


\ivoagroup{Registry}

\author[http://www.ivoa.net/twiki/bin/view/IVOA/PaulHarrison]{Paul Harrison}
\author[http://www.ivoa.net/twiki/bin/view/IVOA/DougBurke]{Douglas Burke}
\author[http://www.ivoa.net/twiki/bin/view/IVOA/RayPlante]{Ray Plante}
\author[http://www.ivoa.net/twiki/bin/view/IVOA/GuyRixon]{Guy Rixon}
\author[http://www.ivoa.net/twiki/bin/view/IVOA/DaveMorris]{Dave Morris}


\editor{Renaud Savalle}

\previousversion[http://www.ivoa.net/Documents/StandardsRegExt/20120217/]{
  PR-20120217}
\previousversion[http://www.ivoa.net/Documents/StandardsRegExt/20120213/]{
  PR-20120213}
\previousversion[http://www.ivoa.net/Documents/StandardsRegExt/20111017/]{
  PR-20111017}
\previousversion[http://www.ivoa.net/Documents/StandardsRegExt/20110921/]{
  PR-20110921}
\previousversion[http://www.ivoa.net/Documents/StandardsRegExt/20110316/]{
  PR-20110316}
\previousversion[http://www.ivoa.net/Documents/StandardsRegExt/20100519/]{
  WD-20100519}

\title{StandardsRegExt}

\begin{document}

\begin{abstract}
This document describes an XML encoding standard for metadata about
IVOA standards themselves, referred to as StandardsRegExt.  It is intended
to allow for the discovery of a standard via an IVOA identifier that
refers to the standard.  It also allows one to define concepts that
are defined by the standard which can themselves be referred to via an
IVOA identifier (augmented with a URL fragment identifier).  Finally,
it can also provide a machine interpretable description of a standard
service interface.  We describe the general model for the schema and
explain its intended use by interoperable registries for discovering
resources.
\end{abstract}


\section*{Conformance-related definitions}

The words ``MUST'', ``SHALL'', ``SHOULD'', ``MAY'', ``RECOMMENDED'', and
``OPTIONAL'' (in upper or lower case) used in this document are to be
interpreted as described in IETF standard RFC2119 \citep{std:RFC2119}.

The \emph{Virtual Observatory (VO)} is a
general term for a collection of federated resources that can be used
to conduct astronomical research, education, and outreach.
The \href{https://www.ivoa.net}{International
Virtual Observatory Alliance (IVOA)} is a global
collaboration of separately funded projects to develop standards and
infrastructure that enable VO applications.

\section*{Acknowledgements}

The first versions of this document have been developed with support from the
National Science Foundation's\footnote{\url{http://www.nsf.gov/}}
Information Technology Research Program under Cooperative Agreement
AST0122449 with The Johns Hopkins University, from the
UK Particle Physics and Astronomy Research Council (PPARC)\footnote{\url{http://www.pparc.ac.uk/}, and from the
European Commission's Seventh Framework
Program\footnote{\url{http://cordis.europa.eu/fp7/capacities/home_en.html}}.

This document contains text lifted verbatim, with small changes, and
with substantial changes from (old version of) the VODataService
specification \citep{.  This
has been done without specific attribution as a means for providing
consistency across similar documents.  We acknowledge the authors of
that document for this text.  



\subsection{Conformance-related definitions}

\label{}


The words "MUST", "SHALL", "SHOULD", "MAY", "RECOMMENDED", and
"OPTIONAL" (in upper or lower case) used in this document are to be
interpreted as described in IETF standard, RFC 2119
\href{#should}{[RFC 2119]}.



The \textbf{Virtual Observatory (VO)} is a
general term for a collection of federated resources that can be used
to conduct astronomical research, education, and outreach.
The\href{http://www.ivoa.net/}{\textbf{International
Virtual Observatory Alliance (IVOA)}} is a global
collaboration of separately funded projects to develop standards and
infrastructure that enable VO applications.



XML document \textbf{validation} is a
software process that checks that an XML document is not only
well-formed XML but also conforms to the syntax rules defined by the
applicable schema.  Typically, when the schema is defined by one or more 
XML Schema [\href{#schema}{Schema}] documents (see
\href{#synnot}{next section}), validation refers to checking for
conformance to the syntax described in those Schema documents.  This
document describes additional syntax constraints that cannot be
enforced solely by the rules of XML Schema; thus, in this document,
use of the term validation includes the extra checks that goes beyond
common Schema-aware parsers which ensure conformance with this
document.  



\subsection{Syntax Notation Using XML Schema}
Syntax Notation Using XML SchemaThe eXtensible Markup Language, or XML, is document syntax for marking
textual information with named tags and is defined by the
World Wide Web Consortium (W3C) Recommendation,
\href{http://www.w3.org/TR/REC-xml}{XML 1.0}
[\href{#xml}{XML}].  The set of XML tag names and the syntax
rules for their use is referred to as the document schema.  One way to
formally define a schema for XML documents is using the W3C standard
known as XML Schema [\href{#schema}{Schema}].


This document defines the StandardsRegExt schema using XML Schema.  The
full Schema document is listed in \href{#appA}{Appendix A}.
Parts of the schema appear within the main sections of this document;
however, documentation nodes have been left out for the sake of brevity.  



Reference to specific elements and types defined in the VOResource
schema include the namespaces prefix, \texttt{vr}, as in
\texttt{vr:Resource} (a type defined in the VOResource schema).
Reference to specific elements and types defined in the StandardsRegExt
schema include the namespaces prefix, \texttt{vstd}, as in
\texttt{vstd:ServiceStandard} (a type defined in the StandardsRegExt schema).
Use of the \texttt{vstd} prefix in compliant instance documents is 
strongly recommended, particularly in the applications that involve
IVOA Registries (see [\href{#r:ri}{RI}], section 3.1.2).
Elsewhere, the use is not required.  



<div>
\section{Contents}

\label{contents}


<div>
<div>
\begin{itemize}

\item \href{#abstract}{Abstract}{}

\item \href{#status}{Status of this document}{}

\item \href{#acknowledge}{Acknowledgments}{}

\item \href{#conf}{Conformance-related definitions}{}

\item \href{#synnot}{Syntax Notation Using XML Schema}{}

\item \href{#Intro}{1. Introduction}
\begin{itemize}

\item \href{#role}{1.1. The Role in the IVOA Architecture}{}

\end{itemize}
{}

\item \href{#model}{2. The StandardsRegExt Data Model}
\begin{itemize}

\item \href{#ns}{2.1. The Schema Namespace and Location}{}

\item \href{#summ}{2.2. Summary of Metadata Concepts}{}

\item \href{#defkeys}{2.3. Defining Enumerations of Identifiers}{}

\end{itemize}
{}

\item \href{#metadata}{3. The StandardsRegExt Metadata}
\begin{itemize}

\item \href{#resext}{3.1.  Resource Type Extensions}{}

\item \href{#stdkey}{3.2.  Defining Keys: StandardKey and StandardKeyURI}{}

\end{itemize}
{}

\item \href{#appA}{Appendix A: The StandardsRegExt XML Schema}{}

\item \href{#appB}{Appendix B: A Sample Record}{}

\item \href{#appC}{Appendix C: Change History}{}

\item \href{#References}{References}{}

\end{itemize}
</div>
</div>
</div>

% - Editor's Note: - Sections are automatically numbered. It should not be necessary - to include section numbers (during the WD stage).

\section{1. Introduction}

\label{Intro}

1. Introduction
An important goal of the IVOA is to publish standards for services
which can interoperate to create a Virtual Observatory (VO).  Central
to the coordination of these services is the concept of a registry
([\href{#r:ri}{RI}]) where resources can be described and thus
discovered by users and applications in the VO.  The standard
\href{http://www.ivoa.net/Documents/REC/ResMetadata/RM-20040426.htm}{Resource
Metadata for the Virtual Observatory} [\href{#RM}{Hanisch et
al. 2004}] (hereafter referred to as \textbf{RM}) defines
metadata terms for services and other discoverable resources.  A
specific XML encoding of these resources is described in the IVOA standard
\href{http://www.ivoa.net/Documents/latest/VOResource.html}{VOResource:
an XML Encoding Schema for Resource Metadata} [\href{#VR}{Plante
et al. 2006}] (hereafter referred to as \textbf{VOResource}).
In this schema, support for a standard service protocol is described
as a service's \emph{capability}; the associated metadata is
contained within the service resource description's
\href{#d:Service}{\texttt{<capability>} element}.  The
specific standard protocol supported is uniquely identified via an
attribute of the \texttt{<capability>} element called
\href{http://www.ivoa.net/Documents/REC/ReR/VOResource-20080222.html#capability-xsi:type}{\texttt{standardID}}
whose value is a URI.  The \href{#VR}{VOResource standard} does
not place a formal validation requirement on the
\texttt{standardID} other than it be a legal URI; however, it
was intended that IVOA-endorsed standards would be represented via an
IVOA identifier.  As per the IVOA Identifier standard
[\href{#ID}{ID}], an IVOA identifier must be registered as a
resource in an IVOA-compliant registry.  



This document defines a VOResource extension schema called
\textbf{StandardsRegExt} which allows one to describe a standard
and register it with an IVOA registry.  By doing so, a unique IVOA
identifier becomes "attached" to the standard which can be referred to
in other resource descriptions, namely for services that support the
standard.  Not only does this aid in the unambiguous discovery of
compliant service instances but also in validating their descriptions
and support for the standard.  Another benefit of associating an IVOA
identifier with a standard is that it allows registry users who discover
services that conform to a particular standard to also discover the
document that describes that standard.



StandardsRegExt has two other purposes.  First, it allows a service
protocol description to communicate specifics about the standard input
parameters and output formats specified by the standard.  Such a
machine-readable description of the interface can assist intelligent
portals and applications to build GUI interfaces to standard services
and manage workflows built around them.  Second, it allows for the
definition of small controlled sets of standardized names (referred to
as \emph{keys} in this document) which might be used to identify,
for example, specific features of a standard protocol (such as
supported data transport protocols).  By virtue of being defined
within the context of a VOResource description, one can refer to the
key using a globally unique URI by adding the key name as a URI fragment
[\href{#r:uri}{URI}] onto the IVOA identifier associated with the
descriptions.  



It is envisaged that StandardsRegExt instances that describe standards
endorsed or otherwise in development by the IVOA will be published in
the IVOA Registry of Registries [\href{#RofR}{RofR}] using the
authority identifier [\href{#ID}{ID}], \texttt{ivoa.net}.
However, other standards, be they ad hoc or endorsed by another
body, may be published in any compliant registry.



\subsection{1.1. The Role in the IVOA Architecture}

\label{}

1.1. The Role in the IVOA Architecture
The IVOA Architecture [\href{#Arch}{Arch}] provides a high-level
view of how IVOA standards work together to connect users and
applications with providers of data and services, as depicted in the
diagram in Fig. 1.  



<center>
<font>
\includegraphics[width=0.9\textwidth]{}
\\
<blockquote>\textbf{Figure 1.  StandardsRegExt in the IVOA Architecture.}
The Registry enables applications in the User Layer to discover
archives and services in the Resource Layer.  The registry metadata
model standards (in blue text and boxes) give structure to the
information that enables that discovery.  StandardsRegExt defines metadata
for describing standards themselves (like those that define the Data
Access Protocols).  
</blockquote>
</font>
</center>



In this architecture, users can leverage a variety of tools (from the
User Layer) to discover archives and services of interest (represented
in the Resource Layer); registries provide the means for this
discovery. A registry is a repository of descriptions of resources
that can be searched based on the metadata in those descriptions. In
general, a resource can be more than just archives, data, or
services; an IVOA standard, as represented by an IVOA document, can
also be a resource.  The Resource Metadata standard
[\href{#RM}{RM}] defines the core concepts used in the resource
descriptions, and the VOResource standard [\href{#VOR}{VOR}] defines
the XML format.  StandardsRegExt is an extension of the VOResource
standard that defines extra metadata for describing a standard.  



\section{2. The StandardsRegExt Data Model}

\label{}

2. The StandardsRegExt Data Model
The StandardsRegExt extension in general enables the description of three
types of resources:

\begin{itemize}

\item  a generic standard (specified by an external document) {}

\item  a standard specifically defining a service protocol {}

\item  a set of related, standardized names called \emph{keys}.  {}

\end{itemize}


Here's an example of defining a controlled list of computer languages
that might be referred to in other descriptions of applications.



<div>
<div>An example of defining a list of
       programming languages </div>

<div>
\begin{verbatim}
<ri:Resource xsi:type="vstd:StandardKeyEnumeration" created="2001-12-31T12:00:00"
             updated="2001-12-31T12:00:00" status="active">
   <title>application languages</title>
   <identifier>ivo://ivoa.net/std/application/languages</identifier>
   <curation>
      <publisher>IVOA</publisher>
      <creator>
         <name>IVOA</name>
         <logo>http://www.ivoa.net/icons/ivoa_logo_small.jpg</logo>
      </creator>
      <date role="representative">2006-07-17</date>
      <version>1.0</version>
      <contact>
         <name>IVOA Grid and Web Services WG</name>
         <email>grid@ivoa.net</email>
      </contact>
   </curation>
   <content>
      <subject>IVOA Standard: registry</subject>
      <description>
         This resource defines keys for commonly used computer languages.
      </description>
      <referenceURL>http://www.ivoa.net/twiki/bin/view/IVOA/IvoaResReg</referenceURL>
   </content>
   <key>
      <name>C</name>
      <description>The C programming language</description>
   </key>
   <key>
      <name>CPP</name>
      <description>The C++ programming language</description>
   </key>
   <key>
      <name>CSharp</name>
      <description>The C# programming language</description>
   </key>
   <key>
      <name>FORTRAN</name>
      <description>The FORTRAN programming language</description>
   </key>
   <key>
      <name>Java</name>
      <description>The Java programming language</description>
   </key>
   <key>
      <name>Perl</name>
      <description>The Perl programming language</description>
   </key>
   <key>
      <name>Python</name>
      <description>The Python programming language</description>
   </key>
</ri:Resource>
\end{verbatim}

</div>
</div>

This description defines the meaning behind the following URI, namely
the Python language:



<blockquote>
\begin{verbatim}
ivo://ivoa.net/std/application/languages#Python
\end{verbatim}

</blockquote>

An application can thus refer to, for example, its support for the
Python language via this URI.  Should other languages become
prevalent, the resource description could be updated to add the new
names, or a new resource description could be created (with a new IVOA
identifier). 



\subsection{2.1. The Schema Namespace and Location}

\label{}

2.1. The Schema Namespace and Location
The namespace associated with StandardsRegExt extensions is
"http://www.ivoa.net/xml/StandardsRegExt/v1.0".  
Just like the namespace URI for the VOResource schema, the
StandardsRegExt namespace URI can be interpreted as a URL.  Resolving it
will return the XML Schema document (given in \href{#appA}{Appendix A})
that defines the StandardsRegExt schema.  



Authors of VOResource instance documents may choose to
provide a location for the VOResource XML Schema document and its
extensions using the
\href{http://www.w3.org/TR/xmlschema-0/#schemaLocation}{\texttt{xsi:schemaLocation}} attribute.  While the choice of
the location value is the choice of the author, this specification
recommends using the StandardsRegExt namespace URI as its location URL
(as illustrated in the example above), as in,



<blockquote>
\begin{verbatim}xsi:schemaLocation="http://www.ivoa.net/xml/StandardsRegExt/v1.0
                    http://www.ivoa.net/xml/StandardsRegExt/v1.0"\end{verbatim}

</blockquote>



<blockquote>
\begin{table}
\begin{tabular}[FIXTHIS]

\begin{description}
\item[\textbf{Note:}] The IVOA Registry Interface standard [\href{#RI}{RI}]
       actually \emph{requires} that the VOResource records it
       shares with other registries provide location URLs via
       \texttt{xsi:schemaLocation} for the VOResource schema and
       all legal extension schemas that are used in the records.  This
       rule would apply to the StandardsRegExt schema.  

\end{description}

\\

\end{tabular}
\end{table}
</blockquote>



The prefix, \texttt{vstd}, is used by convention as the
prefix defined for the StandardsRegExt schema; however, instance documents
may use any prefix of the author's choosing.  In applications where
common use of prefixes is recommended (such as with the Registry
Interface specification [\href{#RI}{RI}]), use of the
\texttt{vstd} prefix is recommended.  Note also that in this
document, the \texttt{vr} prefix is used to label, as shorthand, a
type or element name that is defined in the VOResource schema, as in
\texttt{vr:Resource}. 



As recommend by the VOResource standard [\href{#VR}{VR}], the
StandardsRegExt schema sets \texttt{elementFormDefault="unqualified"}.
This means that it is not necessary to qualify element names defined
in this schema with a namespace prefix (as there are no global
elements defined).  The only place it is usually needed is as a
qualifier to a StandardsRegExt type name given as the value of an
\texttt{xsi:type} attribute.  



\subsection{2.2. Summary of Metadata Concepts}

\label{}

2.2. Summary of Metadata Concepts

The StandardsRegExt extension defines three new types of resources.  Two
are specifically for independently documented standards:


<blockquote>
\begin{description}
\item[\texttt{vstd:Standard}] This resource describes a general standard (e.g. data model,
       schema, protocol, etc.).  The most important piece of metadata
       associated with this resource is the
       \href{http://www.ivoa.net/Documents/REC/ReR/VOResource-20080222.html#content}{\texttt{<referenceURL>}}
       (from the core VOResource schema) 
       which should point to the human-readable specification document
       that defines the standard.  It also allows one to provide the
       recommended version of the standard to use.  
\item[\texttt{vstd:ServiceStandard}] This resource type, which extends from
       \texttt{vstd:Standard}, is specifically for describing a
       standard service protocol (e.g. Simple Cone Search).  It
       differs from \texttt{vstd:Standard} in that it also allows
       one to describe specific constraints on the service interface
       via its
\href{http://www.ivoa.net/Documents/REC/ReR/VOResource-20080222.html#d:interface}{\texttt{<interface>}}
       element. 
\item[\texttt{vstd:StandardKeyEnumeration}] This resource type allows for the description of a related set of 
       controlled names (referred to as \emph{keys}) and their
       meanings.  While keys can be defined as part of a
       \texttt{vstd:Standard} or \texttt{vstd:ServiceStandard}
       resource, the \texttt{vstd:StandardKeyEnumeration} allows 
       a set of key definitions to stand as a resource on its own,
       regardless of whether it is part of a documented standard or
       not.  

\end{description}

</blockquote>

<blockquote>
\begin{table}
\begin{tabular}[FIXTHIS]

\begin{description}
\item[\textbf{Note:}] As mentioned above, this standard allows controlled names to be
       defined either as part of a record of any of the above three
       types.  When such names are being defined as part of an IVOA
       standard, it is recommended that the \texttt{vstd:Standard} or
       \texttt{vstd:ServiceStandard} record corresponding to the
       IVOA standard document be used to define the names.  The
       \texttt{vstd:StandardKeyEnumeration} might be convenient
       for definitions that are experimental or otherwise are for
       non-IVOA-based applications.  

\end{description}

\\

\end{tabular}
\end{table}
</blockquote>

\subsection{2.3. Defining Enumerations of Identifiers }

\label{}

2.3. Defining Enumerations of Identifiers 
A common practice when defining metadata is to restrict
certain string values to a controlled set of defined names, each with 
a well-defined meaning.  With XML
Schema, the controlled set can be enforced by a validating parser 
(using the \texttt{xsd:enumeration} construct
[\href{#schema}{Schema}]).  One disadvantage of locking in the
vocabulary in an XML Schema document is that in order to grow the list
of allowed names, a revision of the XML Schema document is required,
which can be a disruptive change.  To avoid this, it is the practice
within VOResource and its extensions to avoid "hard-coded"
enumerations in the XML Schema document for sets of defined values
that will likely change over time.  



The StandardsRegExt schema provides an alternative to XML Schema-based
definitions of controlled names.  Instead, a controlled list of names,
called \emph{standard keys}, can be defined as part of any of the three
StandardsRegExt resource types.  Updating a resource description is much
less disruptive than a Schema document, and as a resource is available
via an IVOA-compliant registry, it is still possible for a
(non-Schema-based) application to validate the use of the vocabulary.  



The StandardsRegExt specification also defines a mapping from a key name to
a URI.  This allows these keys--and their underlying meaning--to be
referenced in a globally unique way in a variety of contexts, not
restricted to XML.



A key is defined using the \texttt{vstd:StandardKey} type which
consists simply of a name and a description.  The key is mapped to a
URI by attaching the name as the "fragment"--i.e., appending after a
pound sign, \texttt{#}--to the IVOA identifier for the resource
description that defines the key:



<blockquote>\emph{ivoa-identifier}\texttt{#}\emph{key-name}</blockquote>

where \emph{ivoa-identifier} is the resource's IVOA identifier and
\emph{key-name} is the name of a key defined in that resource.
Consistent with the URI standard [\href{#r:uri}{URI}], the
\emph{key-name} must not contain a pound (#) sign.



For example, we consider a resource description with an IVOA
identifier given by \texttt{<identifier>
ivo://ivoa.net/std/QueryProtocol </identifier>} that
defines a a key named \texttt{case-insensitive}; the URI
identifying this key would be: 


<blockquote>\texttt{ivo://ivoa.net/std/QueryProtocol#case-insensitive}</blockquote>



This form of defining multiple keys, each with its own mapping to a
URI, all in one resource has several advantages:



\begin{itemize}

\item  The enumerations are naturally grouped under a single registry
       resource, and so can be retrieved with one registry query and
       need no further metadata to assert the association. {}

\item  The "Dublin core" metadata that is associated with a resource
       need only be entered once for the whole enumeration, rather
       than for each member of the enumeration - this  saves both
       curation effort and space in the registry. {}

\item  If it is necessary to expand the list of controlled names (or
       shrink it), it is simple and fairly undisruptive to update the
       VOResource record.  {}

\end{itemize}

<blockquote>
\begin{table}
\begin{tabular}[FIXTHIS]

\begin{description}
\item[\textbf{Note:}] When these enumerations are presented to a user in a GUI it is
       expected that only the "fragment" part that distinguishes
       the various members of the enumeration will be used as a choice
       value, as the full IVO ID is not usually particularly
       "user-friendly". 

\end{description}

\\

\end{tabular}
\end{table}
</blockquote>

Some applications may wish to publish additional metadata associated
with a key definition through further extension of VOResource
metadata.  This can be be done by deriving a new key metadatum type
derived by extension from the \texttt{vstd:StandardKey}.  



\section{3. The StandardsRegExt Metadata}

\label{}

3. The StandardsRegExt Metadata
\subsection{3.1. Resource Type Extensions}

\label{}

3.1. Resource Type Extensions
This specification defined three new resource types.  As is spelled
out in the VOResource specification, a resource description indicates
that it refers to one of these types of resources by setting the
\texttt{xsi:type} attribute to the namespace-qualified type name.
Doing so implies that the semantic meaning of that type applies to the
resource.  



\subsubsection{3.1.1. Standard}

\label{}

3.1.1. Standard
The \texttt{vstd:Standard} resource type describes any general
standard specification.  This typically refers to an IVOA standard but
is not limited to such.  Generally, the \texttt{vstd:Standard}
type is intended for standards \emph{other than} standard
protocols (which should use the \texttt{vstd:ServiceStandard} type
instead).  It extends the generic \texttt{vr:Resource} type as
follows.  



<div>
<div>vstd:Standard Type Schema Definition</div>

<div>
\begin{verbatim}<xs:complexType name="Standard" >
  <xs:complexContent >
    <xs:extension base="vr:Resource" >
      <xs:sequence >
        <xs:element name="endorsedVersion" type="vstd:EndorsedVersion"
                    maxOccurs="unbounded" />
        <xs:element name="schema" type="vstd:Schema" minOccurs="0" 
                    maxOccurs="unbounded" >
        <xs:element name="deprecated" type="xs:token" minOccurs="0" />
        <xs:element name="key" type="vstd:StandardKey" minOccurs="0"
                    maxOccurs="unbounded" />
      </sequence>
    </extension>
  </complexContent>
</complexType>
\end{verbatim}

</div>
</div>

As one of the purposes of this resource type is to enable users to
discover the documentation that defines the standard that the resource 
describes, the \texttt{<referenceURL>} should point either
to the standard's specification document or to summary information about
the standard that can lead one to the specification document.  



The \texttt{vstd:Standard} resource type adds two metadata terms
to the core set:



\begin{table}
\begin{tabular}[FIXTHIS]

\textbf{vstd:Standard Extension Metadata Elements}\\

\textbf{Element}&
\textbf{Definition}\\
endorsedVersion&
\begin{table}
\begin{tabular}[FIXTHIS]
\emph{Value type:}&a string with optional attributes\\

<tr>
<td>
<em>Semantic Meaning:</em>
</td>

<td>
                     the version of the standard that is recommended for use.
                </td>
</tr>
\emph{Occurrences:}&required; multiple occurrences allowed.\\
\emph{Comments:}&
                     More than one version can be listed, indicating
                     that any of these versions are recommended for
                     use.  
              \\

\end{tabular}
\end{table}
\\
schema&
\begin{table}
\begin{tabular}[FIXTHIS]
\emph{Value type:}&composite: \href{#s:Schema}{vstd:Schema}\\

<tr>
<td>
<em>Semantic Meaning:</em>
</td>

<td>
                     a description and pointer to a schema document
                     defined by this standard.
                </td>
</tr>
\emph{Occurrences:}&optional; multiple occurrences allowed.\\
\emph{Comments:}&
                     This is most typically an XML Schema, but it need
                     not be strictly. 
              \\

\end{tabular}
\end{table}
\\
deprecated&
\begin{table}
\begin{tabular}[FIXTHIS]
\emph{Value type:}&string: 
<code>xs:token</code>
\\
\emph{Semantic Meaning:}&
                     when present, this element indicates that all
                     versions of the standard are considered
                     deprecated by the publisher.  The value is a 
                     human-readable explanation for the designation.
                \\
\emph{Occurrences:}&optional\\
\emph{Comments:}&
                     The explanation should indicate if another
                     standard should be preferred.  
              \\

\end{tabular}
\end{table}
\\
key&
\begin{table}
\begin{tabular}[FIXTHIS]
\emph{Value type:}&composite: 
<code>
<a>vstd:StandardKey</a>
</code>
\\
\emph{Semantic Meaning:}&
                     a defined key associated with this standard.
                \\
\emph{Occurrences:}&optional; multiple occurrences allowed.\\

\end{tabular}
\end{table}
\\

\end{tabular}
\end{table}



The child \texttt{<key>} elements define terms with special
meaning to the standard; see \href{#stdkey}{section 3.2}.  



The purpose of the required \texttt{<endorsedVersion>}
element is to point potential users of the standard to the version
that is most preferred by the standard's publisher.  If multiple
versions are relevant or in use, multiple elements may be given; in
this case, the \texttt{use} attribute can further help steer the
users to the preferred version.



<div>
<div>vstd:EndorsedVersion Type Schema Definition</div>

<div>
\begin{verbatim}<xs:complexType name="EndorsedVersion" >
  <xs:simpleContent >
    <xs:extension base="xs:string" >
      <xs:attribute name="status" default="n/a" >
        <xs:simpleType >
          <xs:restriction base="xs:string" >
            <xs:enumeration value="rec" />
            <xs:enumeration value="pr" />
            <xs:enumeration value="wd" />
            <xs:enumeration value="iwd" />
            <xs:enumeration value="note" />
            <xs:enumeration value="n/a" />
          </restriction>
        </simpleType>
      </attribute>
      <xs:attribute name="use" >
        <xs:simpleType >
          <xs:restriction base="xs:string" >
            <xs:enumeration value="preferred" />
            <xs:enumeration value="deprecated" />
          </restriction>
        </simpleType>
      </attribute>
    </extension>
  </simpleContent>
</complexType>
\end{verbatim}

</div>
</div>

\begin{table}
\begin{tabular}[FIXTHIS]

\textbf{vstd:EndorsedVersion Attributes}\\

\textbf{Attribute}&
\textbf{Definition}\\
status&
\begin{table}
\begin{tabular}[FIXTHIS]
\emph{Value type:}&string with controlled vocabulary\\

<tr>
<td>
<em>Semantic Meaning:</em>
</td>

<td>
                 the IVOA status level of this version of the standard.
               </td>
</tr>
\emph{Occurrences:}&optional\\
\emph{Allowed Values:}&
\begin{table}
\begin{tabular}[FIXTHIS]
\texttt{rec}&
                            an IVOA Recommendation
                        \\
\texttt{pr}&
                            an IVOA Proposed Recommendation
                        \\
\texttt{wd}&
                            an IVOA Working Draft
                        \\
\texttt{iwd}&
                            an internal Working Draft of an IVOA
                            Working Group
                        \\
\texttt{note}&
                            a published IVOA Note
                        \\
\texttt{n/a}&
                            not an IVOA standard or protostandard at
                            this time. 
                        \\

\end{tabular}
\end{table}
\\
\emph{Default Value:}&\texttt{n/a}\\
\emph{Comments:}&
                 For values of \textbf{rec}, \textbf{pr}, 
                 \textbf{wd}, and \textbf{note}, the 
                 record's \textbf{<referenceURL>} element
                 should point to the official specification document in 
                 the \href{http://www.ivoa.net/Documents/}{IVOA
                 Document Repository}; if the document does not
                 appear there, these values should not be used.  
              \\

\end{tabular}
\end{table}
\\
use&
\begin{table}
\begin{tabular}[FIXTHIS]
\emph{Value type:}&string with controlled vocabulary\\

<tr>
<td>
<em>Semantic Meaning:</em>
</td>

<td>
                 A designation of preference for the version compared
                 to other versions in use.
               </td>
</tr>
\emph{Occurrences:}&optional\\
\emph{Allowed Values:}&
\begin{table}
\begin{tabular}[FIXTHIS]
\texttt{preferred}&
                       the most preferred version.
                     \\
\texttt{deprecated}&
                       a version whose use is now discouraged
                       because a newer version is preferred.  
                     \\

\end{tabular}
\end{table}
\\

\end{tabular}
\end{table}
\\

\end{tabular}
\end{table}



When all versions of the standard are considered deprecated by the
resource publisher, the \texttt{<deprecated>} child element
should appear.  The explanation given as a value should mention the
standard that the current standard is deprecated by when relevant.  





<blockquote>
\begin{table}
\begin{tabular}[FIXTHIS]

\begin{description}
\item[\textbf{Note:}] An example where the \texttt{<deprecated>} element
       might be used in the VO is in the case of the SkyNode standard
       [\href{#r:sn}{SkyNode}].  As of this writing, there are
       many instances of SkyNode services available in the VO, and where 
       they are used, version 1.01 is endorsed; however, the IVOA has
       deprecated this standard in favor of the Table Access
       Protocol [\href{#r:tap}{TAP}].  Thus, a
       \textbf{\texttt{vstd:ServiceStandards}} record for
       SkyNode should include a \texttt{<deprecated>}
       element whose content refers viewers to the TAP standard.  

\end{description}

\\

\end{tabular}
\end{table}
</blockquote>

The \textbf{\texttt{<schema>}} element allows one to
list the locations of any schemas defined by the standard thereby making
them discoverable as well (just as the specification document is
discoverable via the \textbf{\texttt{<referenceURL>}}
element).  It also can provide pointers to example uses of the
schemas.  Typically (especially for IVOA standards), the schema is an
XML schema, and the location points to an XML Schema [\href{#schema}{Schema}] document; however, this is not required.
Other schema types and definition formats are allowed.



<div>
<div>vstd:Schema Type Schema Definition</div>

<div>
\begin{verbatim}<xs:complexType name="Schema">
   <xs:sequence>
      <xs:element name="location" type="xs:anyURI" 
                  minOccurs="1" maxOccurs="1"/>
      <xs:element name="description" type="xs:token" 
                  minOccurs="0" maxOccurs="1" />
      <xs:element name="example" type="xs:anyURI" 
                  minOccurs="0" maxOccurs="unbounded" />
   </xs:sequence>
   <xs:attribute name="namespace" type="xs:token" use="required"/>
</xs:complexType>
\end{verbatim}

</div>
</div>

As multiple schemas can be listed in the resource description, the
\textbf{\texttt{namespace}} attribute provides an
identifying label for each \textbf{\texttt{<schema>}} 
element:



\begin{table}
\begin{tabular}[FIXTHIS]

\textbf{vstd:Schema Attributes}\\

\textbf{Attribute}&
\textbf{Definition}\\
namespace&
\begin{table}
\begin{tabular}[FIXTHIS]
\emph{Value type:}&string: \textbf{\texttt{xs:token}}\\

<tr>
<td>
<em>Semantic Meaning:</em>
</td>

<td>
                 an identifier for the schema being described.  Each instance 
                 of this attribute must be unique within the resourse
                 description.
               </td>
</tr>
\emph{Occurrences:}&required\\
\emph{Comments:}&
               For XML schemas, this should be the schema's namespace URI.
               Otherwise, it is a unique label to distinguish it from 
               other schemas described in the same resource description. 
              \\

\end{tabular}
\end{table}
\\

\end{tabular}
\end{table}



The main component of the \textbf{\texttt{<schema>}}
content is the URL pointing to the schema's definition document, but
it can also provide additional information useful for display:



\begin{table}
\begin{tabular}[FIXTHIS]

\textbf{vstd:Schema Metadata Elements}\\

\textbf{Element}&
\textbf{Definition}\\
location&
\begin{table}
\begin{tabular}[FIXTHIS]
\emph{Value type:}&URL: \textbf{\texttt{xs:anyURI}}\\

<tr>
<td>
<em>Semantic Meaning:</em>
</td>

<td>
                 A URL pointing to a document that formally defines
                 the schema.
               </td>
</tr>
\emph{Occurrences:}&required\\
\emph{Comments:}&
                 The document should be in a machine-parsable format
                 when applicable.  For example, when refering to an
                 XML schema, the document should be an XML Schema or 
                 similar document that can be used to validate an 
                 instance document.  
              \\

\end{tabular}
\end{table}
\\
description&
\begin{table}
\begin{tabular}[FIXTHIS]
\emph{Value type:}&URL: \textbf{\texttt{xs:token}}\\

<tr>
<td>
<em>Semantic Meaning:</em>
</td>

<td>
                 A human-readable description of what the schema defines or is 
                 used for.
               </td>
</tr>
\emph{Occurrences:}&optional\\
\emph{Comments:}&
                 A brief description--e.g. one statement--is recommended for
                 display purposes.  
              \\

\end{tabular}
\end{table}
\\
example&
\begin{table}
\begin{tabular}[FIXTHIS]
\emph{Value type:}&URL: \textbf{\texttt{xs:anyURI}}\\

<tr>
<td>
<em>Semantic Meaning:</em>
</td>

<td>
                 A URL pointing to a sample document that illustrates 
                 the use of the schema.
               </td>
</tr>
\emph{Occurrences:}&optional; multiple occurances are allowed.\\
\emph{Comments:}&
                 When applicable (e.g. XML), the document should be
                 in the format defined by the schema document.
              \\

\end{tabular}
\end{table}
\\

\end{tabular}
\end{table}



<div>
<div>An example of a Standard resource that 
       summarizes this specification </div>

<div>
\begin{verbatim}<?xml version="1.0" encoding="UTF-8"?>
<ri:Resource xsi:type="vstd:Standard" status="active"
             created="2012-02-17T11:15:00" updated="2012-02-17T11:15:00"
             xmlns:ri="http://www.ivoa.net/xml/RegistryInterface/v1.0" 
             xmlns:vstd="http://www.ivoa.net/xml/StandardsRegExt/v1.0" 
             xmlns:xsi="http://www.w3.org/2001/XMLSchema-instance">
    <title> StandardsRegExt: a VOResource Schema Extension for Describing IVOA Standards </title>
    <shortName> StandardsRegExt </shortName>
    <identifier>   ivo://ivoa.net/std/StandardsRegExt   </identifier>
    <curation>
      <publisher ivo-id="ivo://ivoa.net/IVOA">
        International Virtual Observatory Alliance
      </publisher>
      <creator>
        <name>  IVOA Registry Working Group   </name>
        <logo>http://www.ivoa.net/icons/ivoa_logo_small.jpg</logo>
      </creator>
      <date role="creation"> 2010-05-19 </date>
      <date role="updated">  2012-02-17 </date>
      <version> 1.0 </version>
      <contact>
        <name>   Resource Registry Working Group   </name>
	<email>  registry@ivoa.net    </email>
      </contact>
    </curation>
    <content>
      <subject>   software standard   </subject>
      <subject>   virtual observatory   </subject>
      <description>
        This document describes an XML encoding standard for metadata about
        IVOA standards themselves, referred to as StandardsRegExt.  It is
        intended to allow for the discovery of a standard via an IVOA
        identifier that refers to the standard.  It also allows one to define
        concepts that are defined by the standard which can themselves be
        referred to via an IVOA identifier (augmented with a URL fragment
        identifier).  Finally, it can also provide a machine interpretable
        description of a standard service interface.  We describe the general
        model for the schema and explain its intended use by interoperable
        registries for discovering resources. 
      </description>
      <referenceURL>
         http://www.ivoa.net/Documents/StandardsRegExt/
      </referenceURL>
      <type>   Other   </type>
      <contentLevel>   Research   </contentLevel>
    </content>
    <endorsedVersion status="pr"> 1.0 </endorsedVersion>
    <schema namespace="http://www.ivoa.net/xml/StandardsRegExt/v1.0">
      <location>http://www.ivoa.net/xml/StandardsRegExt/v1.0</location>
      <description>
        the VOResource extension XML Schema for registering standards
      </description>
      <example>http://rofr.ivoa.net/examples/StandardsRegExt.xml</example>
      <example>http://rofr.ivoa.net/examples/SIA.xml</example>
      <example>http://rofr.ivoa.net/examples/VOSpace.xml</example>
    </schema>
</ri:Resource>
\end{verbatim}

</div>
</div>

\subsubsection{3.1.2. ServiceStandard}

\label{}

3.1.2. ServiceStandard
The \texttt{vstd:ServiceStandard} resource type extends
\href{#d:Standard}{\texttt{vstd:Standard}} to describe more
specifically a standard protocol.  It adds an 
\texttt{<interface>} element to allow the interface defined
by the standard to be described in a machine-readable way.  Its type
is defined to be \texttt{vr:Interface}, which is defined in the
VOResource schema [\href{#r:vr}{VR}].  



<div>
<div>vstd:ServiceStandard Type Schema Definition</div>

<div>
\begin{verbatim}<xs:complexType name="ServiceStandard" >
  <xs:complexContent >
    <xs:extension base="vstd:Standard" >
      <xs:sequence >
        <xs:element name="interface" type="vr:Interface" minOccurs="0"
                    maxOccurs="unbounded" />
      </sequence>
    </extension>
  </complexContent>
</complexType>
\end{verbatim}

</div>
</div>

\begin{table}
\begin{tabular}[FIXTHIS]

\textbf{vstd:ServiceStandard Extension Metadata Elements}\\

\textbf{Element}&
\textbf{Definition}\\
interface&
\begin{table}
\begin{tabular}[FIXTHIS]
\emph{Value type:}&composite: 
<code>vr:Interface</code>
\\
\emph{Semantic Meaning:}&
                      an abstract description of one of the interfaces defined 
                      by this service standard.  
                \\
\emph{Occurrences:}&optional; multiple occurrences allowed.\\
\emph{Comments:}&
               This element can provide details about the interface 
               that apply to all implementations.  Each interface 
               element should specify a \texttt{role} attribute with a 
               value starting with "std:", or, if there is only one standard 
               child element, \texttt{role} is set to "std".  
              \\

\end{tabular}
\end{table}
\\

\end{tabular}
\end{table}



Even though the \texttt{vr:Interface} type requires an
\texttt{<accessURL>} child element, 
the \texttt{<interface>} element in a
\texttt{vstd:ServiceStandard} is intended to describe a service in
the abstract--i.e. without reference to a particular installation of the
service.  Consequently, the accessURL should contain a bogus URL;
applications should not expect it to be resolvable.



An application can, in principle, get a complete machine-readable
description of a particular instance of a standard service (to, say,
create a GUI for that service on-the-fly) by combining the general
description in the \texttt{vstd:ServiceStandard} record with the
service resource description for the specific instance.  The intended
process for building that description is as follows:


\begin{enumerate}

\item  The application obtains a VOResource resource record for the
       service instance (e.g. from a registry).  {}

\item  The application extracts the \texttt{standardID} attribute
       for the desired service capability.  {}

\item  The \texttt{standardID} is resolved (via a registry) to a
       \texttt{vstd:ServiceStandard} record for the service.  This
       description would capture the required and optional (but
       standard) components of the service interface.{}

\item  The specific instance's interface description is merged into
       the standard one.  The service's support of optional components
       as well as its allowed customizations would override the
       generic description from the \texttt{vstd:ServiceStandard}
       record.  {}

\end{enumerate}

The so-called "simple" data access layer (DAL) services, such as the
Simple Image Access services [\href{#r:sia}{SIA}], are
registered using the \texttt{vs:ParamHTTP} interface type
[\href{#r:vds}{VDS}] to describe its interface.  This interface
type allows one to list input parameters accepted by the service.
Each parameter can be marked as \emph{required}, \emph{optional},
or \emph{ignored}.  Typically with DAL services, parameters defined
as optional by the standard may be legally ignored by an
implementation.  Consequently, this specification recommends special
instruction for listing and interpreting input parameteters in a
\texttt{vstd:ServiceStandard} record when the interface is of type
\texttt{vs:ParamHTTP}:  parameters that can be optionally provided
in a client's query but are ignorable by the implementation should be
marked as \emph{ignored}.  Applications that consume such 
\texttt{vstd:ServiceStandard} records should thus interpret the
parameters marked \emph{ignored} as \emph{optional} for use by
clients and \emph{ignorable} by implementations.  This minimizes the
list of parameters that the service provider must list in the
registration of a particular service instance to the ones that are
actually supported (i.e. not ignored): when the list service
instance description is merged into the list from the 
\texttt{vstd:ServiceStandard} record (step 4 above), the result is
an accurate list of the optional but supported and the ignored
parameters for that service instance.



An example of an instance of a \texttt{vstd:ServiceStandard}
record is shown in \href{#appB}{Appendix B}.  It describes the
Simple Image Access Specification [\href{#r:sia}{SIA}] and
in particular illustrates the recommended way to list input parameters
defined by the standard.  



\subsubsection{3.1.3. StandardKeyEnumeration}

\label{}

3.1.3. StandardKeyEnumeration
The \texttt{vstd:StandardKeyEnumeration} resource type is available
for collecting definitions of related, standard keys.  Each key defined
within this resource can then be referred to by a unique IVOA
Identifier URI (see \href{#stdkey}{section 3.2}).  To support
this, the \texttt{vstd:StandardKeyEnumeration} resource simply
adds the \texttt{<key>} element to the standard core
metadata.  



<div>
<div>vstd:StandardKeyEnumeration Type Schema Definition</div>

<div>
\begin{verbatim}<xs:complexType name="StandardKeyEnumeration" >
  <xs:complexContent >
    <xs:extension base="vr:Resource" >
      <xs:sequence >
        <xs:element name="key" type="vstd:StandardKey" maxOccurs="unbounded"
                    minOccurs="1" />
      </sequence>
    </extension>
  </complexContent>
</complexType>
\end{verbatim}

</div>
</div>

\begin{table}
\begin{tabular}[FIXTHIS]

\textbf{vstd:StandardKeyEnumeration Extension Metadata Elements}\\

\textbf{Element}&
\textbf{Definition}\\
key&
\begin{table}
\begin{tabular}[FIXTHIS]
\emph{Value type:}&composite: 
<code>
<a>vstd:StandardKey</a>
</code>
\\
\emph{Semantic Meaning:}&
                      the name and definition of a key--a named concept, 
                      feature, or property.
                \\
\emph{Occurrences:}&required; multiple occurrences allowed.\\

\end{tabular}
\end{table}
\\

\end{tabular}
\end{table}



The contents of the \texttt{<key>} element is described in
the next section.  



\subsection{3.2. Defining Keys: StandardKey and StandardKeyURI}

\label{}

3.2. Defining Keys: StandardKey and StandardKeyURI
The \texttt{vstd:StandardKey} type provides the means to 
define keys (as defined in \href{#sec:keys}{section 2.3}) within
a VOResource record.



<div>
<div>vstd:StandardKey Type Schema Definition</div>

<div>
\begin{verbatim}<xs:complexType name="StandardKey" >
  <xs:sequence >
    <xs:element name="name" type="vstd:fragment" />
    <xs:element name="description" type="xs:token" />
  </sequence>
</complexType>

<xs:simpleType name="fragment" >
  <xs:restriction base="xs:string" >
    <xs:pattern value="([A-Za-z0-9;/\?:@&amp;=\+$,\-_.!~\*'\(\)]|%[A-Fa-f0-9]{2})+" />
  </restriction>
</simpleType>
\end{verbatim}

</div>
</div>

\begin{table}
\begin{tabular}[FIXTHIS]

\textbf{vstd:StandardKey Metadata Elements}\\

\textbf{Element}&
\textbf{Definition}\\
name&
\begin{table}
\begin{tabular}[FIXTHIS]
\emph{Value type:}&
               string with a form restricted to a legal URI fragment
               [\href{#r:uri}{URI}]. 
              \\
\emph{Semantic Meaning:}&
                  The property identifier which would appear as the
                  fragment (string after the pound sign, #) in an IVOA
                  identifier.  
               \\
\emph{Occurrences:}&required\\
\emph{Comments:}&
                Note that fragments may not include a pound (#) sign.
              \\

\end{tabular}
\end{table}
\\
description&
\begin{table}
\begin{tabular}[FIXTHIS]
\emph{Value type:}&string: 
<code>xs:token</code>
\\
\emph{Semantic Meaning:}&
                  A human-readable definition of this property.  
               \\
\emph{Occurrences:}&required\\
\emph{Comments:}&
                Note that while a lengthy definition and other
                comments can be included in the body of this element,
                the \texttt{xs:token} type does not support
                multiple paragraphs.  
              \\

\end{tabular}
\end{table}
\\

\end{tabular}
\end{table}



Defining a key via a \texttt{<key>} element within a
VOResource record implies the definition of a unique URI formed
according to the syntax described in \href{#sec:keys}{section
2.3} that represents the semantics given by the value of the 
\texttt{<description>} child element.  Because the URI must
be globally unique, the key name (given by the
\texttt{<name>} child element) must be unique within the
VOResource record.



Though it is not needed by StandardsRegExt resource records, the StandardsRegExt
schema further defines a convenience type,
\texttt{vstd:StandardKeyURI}, which defines the legal pattern for
a full standard key identifier (as defined in
\href{#sec:keys}{section 2.3}).  Applications that wish to use
XML Schema to validate the form of a key URI may import the StandardsRegExt
schema and use this type.  





<blockquote>
\begin{table}
\begin{tabular}[FIXTHIS]

\begin{description}
\item[\textbf{Note:}] It is worth noting that just using or otherwise referencing a
       standard key URI in an application does not require importing
       the StandardsRegExt nor need there be any reference to the
       StandardsRegExt namespace.  The role of the StandardsRegExt schema is
       simply to provide a way of documenting the definitions in a
       VOResource record.  Thus, an application may dereference the
       URI for display or user help purposes; however, dereferencing
       is not necessary to use the URI. 

\end{description}

\\

\end{tabular}
\end{table}
</blockquote>

\section{Appendix A: The complete StandardsRegExt Schema}

\label{}

Appendix A: The complete StandardsRegExt SchemaThe schema is included here for completeness - the definitive source of the schema is at \href{http://www.ivoa.net/xml/StandardsRegExt/v1.0}{http://www.ivoa.net/xml/StandardsRegExt/v1.0}.


<div>
<div>The Complete StandardsRegExt Schema</div>

<div>
\begin{verbatim}<?xml version="1.0" encoding="UTF-8"?>
<xs:schema targetNamespace="http://www.ivoa.net/xml/StandardsRegExt/v1.0" 
           xmlns:xs="http://www.w3.org/2001/XMLSchema" 
           xmlns:vr="http://www.ivoa.net/xml/VOResource/v1.0" 
           xmlns:vstd="http://www.ivoa.net/xml/StandardsRegExt/v1.0" 
           xmlns:vm="http://www.ivoa.net/xml/VOMetadata/v0.1"
           elementFormDefault="unqualified" attributeFormDefault="unqualified"
           version="1.0" >

   <xs:annotation>
      <xs:appinfo>
        <vm:schemaName>StandardsRegExt</vm:schemaName>
        <vm:schemaPrefix>xs</vm:schemaPrefix>
        <vm:targetPrefix>vstd</vm:targetPrefix>
      </xs:appinfo>
      <xs:documentation>
        This is a core schema for describing IVOA Standards themselves
      </xs:documentation>
   </xs:annotation>

   <xs:import namespace="http://www.ivoa.net/xml/VOResource/v1.0" 
              schemaLocation="http://www.ivoa.net/xml/VOResource/v1.0"/>

   <xs:complexType name="Standard">
      <xs:annotation>
         <xs:documentation>
           a description of a standard specification.
         </xs:documentation>
         <xs:documentation>
           This typically refers to an IVOA standard but is not
           limited to such.  
         </xs:documentation>
      </xs:annotation>

      <xs:complexContent>
         <xs:extension base="vr:Resource">
            <xs:sequence>
 
              <xs:element name="endorsedVersion" type="vstd:EndorsedVersion" 
                          maxOccurs="unbounded">
                <xs:annotation>
                   <xs:documentation>
                     the version of the standard that is recommended for use.
                   </xs:documentation>
                   <xs:documentation>
                     More than one version can be listed, indicating
                     that any of these versions are recognized as
                     acceptable for use.  
                   </xs:documentation>
                </xs:annotation>
              </xs:element>

              <xs:element name="schema" type="vstd:Schema" 
                          minOccurs="0" maxOccurs="unbounded">
                <xs:annotation>
                   <xs:documentation>
                     a description and pointer to a schema document
                     defined by this standard.
                   </xs:documentation>
                   <xs:documentation>
                     This is most typically an XML Schema, but it need
                     not be strictly.  
                   </xs:documentation>
                </xs:annotation>
              </xs:element>

              <xs:element name="deprecated" type="xs:token" minOccurs="0">
                <xs:annotation>
                   <xs:documentation>
                     when present, this element indicates that all
                     versions of the standard are considered
                     deprecated by the publisher.  The value is a 
                     human-readable explanation for the designation.
                   </xs:documentation>
                   <xs:documentation>
                     The explanation should indicate if another
                     standard should be preferred.  
                   </xs:documentation>
                </xs:annotation>
              </xs:element>

              <xs:element name="key" type="vstd:StandardKey" 
                          minOccurs="0" maxOccurs="unbounded">
                <xs:annotation>
                   <xs:documentation>
                     a defined key associated with this standard.
                   </xs:documentation>
                </xs:annotation>
              </xs:element>

            </xs:sequence>
         </xs:extension>
      </xs:complexContent>
   </xs:complexType>

   <xs:complexType name="EndorsedVersion">
      <xs:simpleContent>
         <xs:extension base="xs:string">
           <xs:attribute name="status" default="n/a">
             <xs:annotation>
               <xs:documentation>
                 the IVOA status level of this version of the standard.
               </xs:documentation>
               <xs:documentation>
                 For values of "rec", "pr", "wd", and "note", the
                 record's referenceURL element should point to the
                 official specification document in the IVOA Document
                 Repository; if the document does not appear there,
                 these values should not be used. 
               </xs:documentation>
             </xs:annotation>

             <xs:simpleType>
                <xs:restriction base="xs:string">
                   <xs:enumeration value="rec">
                      <xs:annotation>
                         <xs:documentation>
                            an IVOA Recommendation
                         </xs:documentation>
                      </xs:annotation>
                   </xs:enumeration>
                   <xs:enumeration value="pr">
                      <xs:annotation>
                         <xs:documentation>
                            an IVOA Proposed Recommendation
                         </xs:documentation>
                      </xs:annotation>
                   </xs:enumeration>
                   <xs:enumeration value="wd">
                      <xs:annotation>
                         <xs:documentation>
                            an IVOA Working Draft
                         </xs:documentation>
                      </xs:annotation>
                   </xs:enumeration>
                   <xs:enumeration value="iwd">
                      <xs:annotation>
                         <xs:documentation>
                            an internal Working Draft of an IVOA Working Group
                         </xs:documentation>
                      </xs:annotation>
                   </xs:enumeration>
                   <xs:enumeration value="note">
                      <xs:annotation>
                         <xs:documentation>
                            a published IVOA Note
                         </xs:documentation>
                      </xs:annotation>
                   </xs:enumeration>
                   <xs:enumeration value="n/a">
                      <xs:annotation>
                         <xs:documentation>
                            not an IVOA standard or protostandard at
                            this time. 
                         </xs:documentation>
                      </xs:annotation>
                   </xs:enumeration>
                </xs:restriction>
             </xs:simpleType>
           </xs:attribute>

           <xs:attribute name="use">
             <xs:annotation>
               <xs:documentation>
                 A designation of preference for the version compared
                 to other versions in use.
               </xs:documentation>
             </xs:annotation>

             <xs:simpleType>
                <xs:restriction base="xs:string">
                   <xs:enumeration value="preferred">
                      <xs:annotation>
                         <xs:documentation>
                            the most preferred version.
                         </xs:documentation>
                         <xs:documentation>
                            Only one version should have this designation.
                         </xs:documentation>
                      </xs:annotation>
                   </xs:enumeration>
                   <xs:enumeration value="deprecated">
                      <xs:annotation>
                         <xs:documentation>
                            a version whose use is now discouraged
                            because a newer version is preferred.  
                         </xs:documentation>
                      </xs:annotation>
                   </xs:enumeration>
                </xs:restriction>
             </xs:simpleType>
           </xs:attribute>

         </xs:extension>
      </xs:simpleContent>
   </xs:complexType>

   <xs:complexType name="Schema">
      <xs:annotation>
         <xs:documentation>
           a description of a schema definition
         </xs:documentation>
      </xs:annotation>

      <xs:sequence>
         <xs:element name="location" type="xs:anyURI" 
                     minOccurs="1" maxOccurs="1">
            <xs:annotation>
               <xs:documentation>
                  A URL pointing to a document that formally defines
                  the schema.
               </xs:documentation>
               <xs:documentation>
                  The document should be in a machine-parsable format
                  when applicable.  For example, when refering to an
                  XML schema, the document should be an XML Schema or 
                  similar document that can be used to validate an 
                  instance document.  
               </xs:documentation>
            </xs:annotation>
         </xs:element>
         <xs:element name="description" type="xs:token" 
                     minOccurs="0" maxOccurs="1" >
            <xs:annotation>
               <xs:documentation>
                  A human-readable description of what the schema
                  defines or is used for.
               </xs:documentation>
               <xs:documentation>
                  A brief description--e.g. one statement--is
                  recommended for display purposes.  
               </xs:documentation>
            </xs:annotation>
         </xs:element>
         <xs:element name="example" type="xs:anyURI" 
                     minOccurs="0" maxOccurs="unbounded">
            <xs:annotation>
               <xs:documentation>
                  A URL pointing to a sample document that illustrates 
                  the use of the schema.
               </xs:documentation>
               <xs:documentation>
                  When applicable (e.g. XML), the document should be
                  in the format defined by the schema document.
               </xs:documentation>
            </xs:annotation>
         </xs:element>
      </xs:sequence>

      <xs:attribute name="namespace" type="xs:token" use="required">
          <xs:annotation>
             <xs:documentation>
               an identifier for the schema being described.  Each instance 
               of this attribute must be unique within the resourse description.
             </xs:documentation>
             <xs:documentation>
               For XML schemas, this should be the schema's namespace URI.
               Otherwise, it should be a unique label to distinguish it from 
               other schemas described in the same resource description. 
             </xs:documentation>
          </xs:annotation>
      </xs:attribute>
   </xs:complexType>

   <xs:complexType name="ServiceStandard">
      <xs:annotation>
         <xs:documentation>
           a description of a standard service protocol.
         </xs:documentation>
         <xs:documentation>
           This typically refers to an IVOA standard but is not
           limited to such.  
         </xs:documentation>
      </xs:annotation>

      <xs:complexContent>
         <xs:extension base="vstd:Standard">
            <xs:sequence>
               <xs:element name="interface" type="vr:Interface"
                           minOccurs="0" maxOccurs="unbounded">
                  <xs:annotation>
                    <xs:documentation>
                      an abstract description of one of the interfaces defined 
                      by this service standard.  
                    </xs:documentation>
                    <xs:documentation>
                      This element can provide details about the interface 
                      that apply to all implementations.  Each interface 
                      element should specify a role with a value starting 
                      with "std:" or, if there is only one standard interface,
                      is equal to "std".  
                    </xs:documentation>
                    <xs:documentation>
                      Applications that, for example, wish to build a GUI
                      to the service on-the-fly would first access this generic 
                      description.  Site-specific variations, such
                      as supported optional arguments or site specific 
                      arguments, would be given in the actual implemented 
                      service description and overrides any common information 
                      found in this generic description.  This generic interface
                      description can be matched with the site-specific one 
                      using the role attribute.  
                    </xs:documentation>
                    <xs:documentation>
                      Even though the Interface type requires an
                      accessURL child element, this element is
                      intended to describe a service in the
                      abstract--i.e. without reference to a particular 
                      installation of the service.  Consequently,
                      the accessURL may contain a bogus URL;
                      applications should not expect it to be resolvable.  
                    </xs:documentation>
                  </xs:annotation>
               </xs:element>
            </xs:sequence>
         </xs:extension>
      </xs:complexContent>
   </xs:complexType>

   <xs:complexType name="StandardKeyEnumeration">
      <xs:annotation>
         <xs:documentation>
            A registered set of related keys.  Each key can be
            uniquely identified by combining the IVOA identifier of
            this resource with the key name separated by the URI
            fragment delimiter, #, as in: ivoa-identifier#key-name
         </xs:documentation>
      </xs:annotation>
      <xs:complexContent>
         <xs:extension base="vr:Resource">
            <xs:sequence>
               <xs:element name="key" type="vstd:StandardKey"
                           maxOccurs="unbounded" minOccurs="1">
                 <xs:annotation>
                    <xs:documentation>
                      the name and definition of a key--a named concept, 
                      feature, or property.
                    </xs:documentation>
                 </xs:annotation>
               </xs:element>
            </xs:sequence>
         </xs:extension>
      </xs:complexContent>
   </xs:complexType>

   <xs:complexType name="StandardKey">
      <xs:annotation>
         <xs:documentation>
            The name and definition of a key--a named concept, 
            feature, or property.
         </xs:documentation>
         <xs:documentation>
            This key can be identified via an IVOA identifier
            of the form ivo://authority/resource#name where name is
            the value of the child name element.
         </xs:documentation>
         <xs:documentation>
            This type can be extended if the key has
            other metadata associated with it. 
         </xs:documentation>
      </xs:annotation>

      <xs:sequence>
         <xs:element name="name" type="vstd:fragment">
            <xs:annotation>
               <xs:documentation>
                  The property identifier which would appear as the
                  fragment (string after the pound sign, #) in an IVOA
                  identifier.  
               </xs:documentation>
            </xs:annotation>
         </xs:element>
         <xs:element name="description" type="xs:token">
            <xs:annotation>
               <xs:documentation>
                  A human-readable definition of this property.  
               </xs:documentation>
            </xs:annotation>
         </xs:element>
      </xs:sequence>
   </xs:complexType>

   <xs:simpleType name="StandardKeyURI">
      <xs:annotation>
         <xs:documentation>
            reference that forces an IVOA ID with a # fragment part
            appended to match the standard pattern for registering
            enumeration values with a vstd:StandardKeyList
         </xs:documentation>
      </xs:annotation>
      <xs:restriction base="xs:anyURI">
         <xs:pattern value="ivo://[\w\d][\w\d\-_\.!~\*'\(\)\+=]{2,}(/[\w\d\-_\.!~\*'\(\)\+=]+(/[\w\d\-_\.!~\*'\(\)\+=]+)*)?(#([A-Za-z0-9;/\?:@&amp;=\+$,\-_\.!~\*'\(\)]|%[A-Fa-f0-9]{2})+)?"/>
      </xs:restriction>
    </xs:simpleType>

    <xs:simpleType name="fragment">
       <xs:annotation>
          <xs:documentation>
             the allowed characters for a fragment identifier taken
             from rfc2396
          </xs:documentation>
       </xs:annotation>
       <xs:restriction base="xs:string">
          <xs:pattern
              value="([A-Za-z0-9;/\?:@&amp;=\+$,\-_\.!~\*'\(\)]|%[A-Fa-f0-9]{2})+"/>
       </xs:restriction>
    </xs:simpleType>
</xs:schema>
\end{verbatim}

</div>
</div>

\section{Appendix B: A Sample Record}

\label{}

Appendix B: A Sample Record
This example shows how one can describe an IVOA standard, the Simple
Image Access Protocol.  It includes a description of the input parameters
defined in the specification.  



<div>
<div>A
  sample record describing the SIA standard</div>

<div>
\begin{verbatim}><?xml version="1.0" encoding="UTF-8"?>
<resource xsi:type="vstd:ServiceStandard" status="active"
          created="2000-01-01T09:00:00" updated="2000-01-01T09:00:00"
          xmlns:vstd="http://www.ivoa.net/xml/StandardsRegExt/v1.0" 
          xmlns:vs="http://www.ivoa.net/xml/VODataService/v1.1" 
          xmlns:xsi="http://www.w3.org/2001/XMLSchema-instance"
          xsi:schemaLocation="http://www.ivoa.net/xml/VOResource/v1.0
                              VOResource-v1.0.xsd
                              http://www.ivoa.net/xml/VODataService/v1.1
                              VODataService-v1.1.xsd
                              http://www.ivoa.net/xml/StandardsRegExt/v1.0
                              StandardsRegExt-v1.0.xsd">

    <title>   Simple Image Access Protocol   </title>
    <shortName>   SIA   </shortName>
    <identifier>   ivo://ivoa.net/std/SIA   </identifier>

    <curation>
      <publisher ivo-id="ivo://ivoa.net/IVOA">
        International Virtual Observatory Alliance
      </publisher>
      <creator>
        <name>  Doug Tody   </name>
      </creator>
      <creator>
        <name>  Ray Plante   </name>
      </creator>
      <date>   2004-05-24   </date>
      <contact>
        <name>   Data Access Layer Working Group   </name>
	<email>  dal@ivoa.net    </email>
      </contact>
    </curation>

    <content>
      <subject>   software standard   </subject>
      <subject>   virtual observatory   </subject>
      <description>
        The Simple Image Access Protocol is a protocol for retrieving
        image data from a variety of astronomical image repositories
        through a uniform interface. The interface is meant to be
        reasonably simple to implement by service providers. A query
        defining a rectangular region on the sky is used to query for
        candidate images. The service returns a list of candidate
        images formatted as a VOTable. For each candidate image an
        access reference URL may be used to retrieve the image. Images
        may be returned in a variety of formats including FITS and
        various graphics formats. Referenced images are often computed
        on the fly, e.g., as cutouts from larger images. 
      </description>
      <referenceURL>
         http://www.ivoa.net/Documents/latest/SIA.html
      </referenceURL>
      <type>   Other   </type>
      <contentLevel>   Research   </contentLevel>
    </content>

    <endorsedVersion status="rec"> 1.0 </endorsedVersion>

    <interface xsi:type="vs:ParamHTTP" role="std" version="1.0">
       <accessURL use="base"> http://sample.org/cgi-bin/sia </accessURL>
       <queryType>GET</queryType>
       <resultType>text/xml+votable</resultType>

       <!-- 
         -  These are the standard input parameters defined in the 
         -  SIA spec
         -->
       <param use="required">
         <name>POS</name>
         <description>
            Search Position in the form "ra,dec" where ra and dec are given 
            in decimal degrees in the ICRS coordinate system.
         </description>
         <unit>degrees</unit>
         <dataType arraysize="2">real</dataType>
       </param>
       <param use="required">
         <name>SIZE</name>
         <description>
            Size of search region in the RA and Dec. directions in decimal 
            degrees.
         </description>
         <unit>degrees</unit>
         <dataType arraysize="2">real</dataType>
       </param>
       <param use="optional">
         <name>FORMAT</name>
         <description>Requested format of images.</description>
         <dataType>string</dataType>
       </param>
       <param use="optional">
         <name>INTERSECT</name>
         <description>
           Choice of intersection of matched images with the region of 
           interest.
         </description>
         <dataType>string</dataType>
       </param>
       <param use="ignored">
         <name>NAXIS</name>
         <description>
           The number of pixels desired along each axis
         </description>
         <dataType arraysize="2">integer</dataType>
       </param>
       <param use="ignored">
         <name>CFRAME</name>
         <description>
           the coordinate frame to impose on the image.
         </description>
         <dataType>string</dataType>
       </param>
       <param use="ignored">
         <name>EQUINOX</name>
         <description>
           the epoch of the mean equator and equinox for the specified 
           coordinate system reference frame (CFRAME). coordinate frame to 
           impose on the image.
         </description>
         <dataType>string</dataType>
       </param>
       <param use="ignored">
         <name>CRPIX</name>
         <description>
           the pixel position to locate the reference position in the output
           image.
         </description>
         <dataType arraysize="2">real</dataType>
       </param>
       <param use="ignored">
         <name>CRVAL</name>
         <description>
           the world coordinates of the reference position in the output
           image in decimal degrees
         </description>
         <dataType arraysize="2">real</dataType>
       </param>
       <param use="ignored">
         <name>CDELT</name>
         <description>
           the scale of the output image in decimal degrees per pixel
         </description>
         <dataType arraysize="2">real</dataType>
       </param>
       <param use="ignored">
         <name>ROTANG</name>
         <description>
           the rotation angle to put into the output image's coordinate system
         </description>
         <dataType arraysize="2">real</dataType>
       </param>
       <param use="ignored">
         <name>PROJ</name>
         <description>
           the projection to impose in the construction of the output image
         </description>
         <dataType>string</dataType>
       </param>
       <param use="ignored">
         <name>VERB</name>
         <description>
           the level of verbosity in the output.
         </description>
         <dataType>string</dataType>
       </param>
    </interface>
   
</resource>
\end{verbatim}

</div>
</div>

\section{Appendix C: Change History}

\label{}

Appendix C: Change History
\subsubsection{Changes since PR-v1.0 20120217:}
Changes since PR-v1.0 20120217:
\begin{itemize}

\item  none other than date and status.{}

\end{itemize}

\subsubsection{Changes since PR-v1.0 20120213:}
Changes since PR-v1.0 20120213:
\begin{itemize}

\item  added the \textbf{\texttt{<schema>}} element to
       \textbf{\texttt{vstd:Standard}}{}

\item  updated example in section 3.1.1 {}

\end{itemize}

\subsubsection{Changes since PR-v1.0 20111017:}
Changes since PR-v1.0 20111017:
\begin{itemize}

\item  added Note box to section 2.2 that recommends against using 
       \textbf{\texttt{vstd:StandardKeyEnumeration}} to
       describe keys when they are defined by an IVOA standard. {}

\item  added statement in section 2.3 highlighting that # signs are
       not allowed in key names. {}

\item  added \texttt{iwd} and \texttt{note} as allowed values
       for \textbf{\texttt{vstd:EndorsedVersion}}'s
       \textbf{\texttt{status}} attribute. {}

\item  converted a Note box in section 3.1.2 to a normative paragraph
       that recommends listing optional \textbf{\texttt{ParamHTTP}}
       parameters as ignored.  Note that there is a related error in
       the VODataService standard [\href{#r:vds}{VDS}]: while
       the \textbf{\texttt{ignored}} value is defined in
       the schema, it is not included in the document text.{}

\item  added sample \textbf{\texttt{vstd:ServiceStandard}} 
       instance record back in as Appendix B {}

\item  added references to TAP and SIA {}

\item  fixed various grammatical typos. {}

\end{itemize}

\subsubsection{Changes since PR-v1.0 20110921:}
Changes since PR-v1.0 20110921:
\begin{itemize}

\item  In \texttt{<endorsedVersion>}, changed "prop" to "pr". {}

\item  various typo corrections {}

\end{itemize}

\subsubsection{Changes since PR-v1.0 20110316:}
Changes since PR-v1.0 20110316:
\begin{itemize}

\item Corrected ampersand representation in schema listing{}

\item Various typo corrections and clarifications{}

\end{itemize}

\subsubsection{Changes since WD-v1.0 20100519:}
Changes since WD-v1.0 20100519:
\begin{itemize}

\item  Prepped for PR {}

\item  improved discussion of example in section 2 {}

\item  Added standard architecture sub-section {}

\item  updated in-lined schema (App. 1) {}

\end{itemize}

\subsubsection{Changes since WD-v1.0 20100514:}
Changes since WD-v1.0 20100514:
\begin{itemize}

\item  short name changed from VOStandard to StandardsRegExt {}

\end{itemize}

\subsubsection{Changes since WD-v0.4:}
Changes since WD-v0.4:
\begin{itemize}

\item  removed App. B. (Sample instance) as examples appear throughout
       the document. {}

\end{itemize}

\section{References}

\label{References}

References
<hr></hr>

$Revision: 1636 $ $Date: 2012-02-17 08:53:29 -0600 (Fri, 17 Feb 2012) $  $HeadURL: https://volute.googlecode.com/svn/trunk/projects/registry/StandardsRegExt/StandardsRegExt-v1.0pr.html $


</div>

% end id="main"

\end{document}
